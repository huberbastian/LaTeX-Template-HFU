\documentclass[
    12pt,                              % Schriftgröße
    fleqn,                             % Gleichungen linksbündig
    twoside,
    headsepline,                       % einzeichnen einer Linie unter die Kopfzeile
    listof=totoc,
    toc=chapterentrywithdots,
    bibliography=totocnumbered,        % Nummerierung des Literaturverzeichnis
    parskip=half,                      % Absatzabstände erzeugen
]{scrreprt}

\usepackage[                            % Seitenränder
    left = 25mm,
    right = 25mm,
    top = 30mm,
    bottom = 25mm,
    bindingoffset=15mm,
]{geometry}

\usepackage[pdfpagelabels,
    pdfstartview = FitH,
    bookmarksopen = false,
    bookmarksnumbered = true,
    linkcolor = black,
    plainpages = false,
    hypertexnames = false,
    citecolor = black
]{hyperref}

\newcommand*{\captionsource}[2]{%
  \caption[{#1}]{%
    #1%
    \\\hspace{\linewidth}%
    Quelle: #2%
  }%
}


\usepackage[T1]{fontenc}
\usepackage[utf8]{inputenc}                 % UTF8 (Windowsspezifisch), Für Umlaute wie ä,ö,ü,$,...
\usepackage[ngerman]{babel}                 % Deutsch
\usepackage{lmodern}                        % Schriftart 
\usepackage{graphicx, subfig}
\usepackage{float}
\usepackage{scrhack}                        % Warnung wegen floats unterdrücken
\usepackage{pdfpages}
\graphicspath{{img/}}
\setkomafont{disposition}{\bfseries}        % Dei Schriftart für alles setzen (Überschriften etc.)
\usepackage[onehalfspacing]{setspace}       % Zeilenabstand 1,5cm
\BeforeStartingTOC[toc]{\singlespacing}     % "Korrekter" Zeilenabstand im Inhaltsverzeichnis

\usepackage{scrlayer-scrpage}               % Um Kopf- und Fußzeile ändern zu können
\clearpairofpagestyles                      % Löscht Standardeinstellungen

\ihead{\normalfont{\headmark}}                  % Header Inside
\automark[chapter]{chapter}                     % \automark[rechte Seite]{linke Seite}
\chead{}                                        % Header Center
\ohead{\pagemark}                               % Header Outside
\renewcommand*\chapterpagestyle{scrheadings}    % Header auch für \chapter generieren
    % Wie oben nur für Footer
\ifoot{\normalfont\footnotesize{Bastian Huber}}
\cfoot{}
\ofoot{\normalfont\footnotesize{\today}}

\usepackage{lipsum}                         % Blindtext mit \lipsum generieren
                    
\addtokomafont{chapter}{\large}             % Chapter, Section etc. Schriftgröße
\addtokomafont{section}{\normalfont\large}
\addtokomafont{subsection}{\normalfont\normalsize}

%#################################################################
% Begin des Document
% ----------------------------------------------------------------
\begin{document}
    \pagenumbering{Roman}
    %#################################################################
    % Titelseite
    % ----------------------------------------------------------------
    \input{1_Titelseite/Titelseite.tex}  
    \cleardoublepage
    
    %#################################################################
    % Abstract und Vorwort
    % ----------------------------------------------------------------
    \addchap{Abstract}
\lipsum[1]
    \cleardoublepage

    %#################################################################
    % ToC
    % ----------------------------------------------------------------
    \tableofcontents
    \cleardoublepage 

    \listoffigures
    \cleardoublepage 

    %\listoftables
    %\cleardoublepage                  % zum Beenden der roman Seitennummerierung! \clearpage

    \pagenumbering{arabic}            % Seitennummerierung Hauptteil
    %#################################################################
    % Inhalt
    % ----------------------------------------------------------------                      
    \include{3_Inhalt/Einleitung.tex}
    \cleardoublepage
    \include{3_Inhalt/Hauptteil.tex}
    \cleardoublepage
    \include{3_Inhalt/Schluss.tex}
    \cleardoublepage

    %#################################################################
    % Anhaenge
    % ----------------------------------------------------------------                      
    \include{4_Anhaenge/Eidesstattliche_Erklaerung.tex}

    %\raggedright                       % Blocksatz ausstellen
    %\bibliographystyle{unsrt}           % Literatur nummeriert darstellen
    \bibliographystyle{plain}
    \bibliography{Literaturstellen}     % .bib einbinden

\end{document}
